%%%%%%%%%%%%%%%%%%%%%%%%%%%%%%%%%%%%%%%%%
% Long Professional Curriculum Vitae
% LaTeX Template
% Version 1.1 (9/12/12)
%
% This template has been downloaded from:
% http://www.latextemplates.com
% Original author:
% Rensselaer Polytechnic Institute (http://www.rpi.edu/dept/arc/training/latex/resumes/)
%
% Important note:
% This template requires the res.cls file to be in the same directory as the
% .tex file. The res.cls file provides the resume style used for structuring the
% document.
%
%%%%%%%%%%%%%%%%%%%%%%%%%%%%%%%%%%%%%%%%%

%----------------------------------------------------------------------------------------
%	PACKAGES AND OTHER DOCUMENT CONFIGURATIONS
%----------------------------------------------------------------------------------------
%\let\nofiles\relax
\documentclass{my_cv} 

\usepackage{hyperref}
\usepackage[letterpaper,bindingoffset=0.2in,left=0.75in,right=0.75in,top=0.5in,bottom=1in,footskip=0.4in]{geometry}

\usepackage{marvosym}
\usepackage{fontawesome}
\usepackage{academicons}
\usepackage[document]{ragged2e}

\usepackage[backend=bibtex,
style=numeric,
style=alphabetic,
bibencoding=ascii,
defernumbers=true,
style=ieee,
sorting=nyt
]{biblatex}
\addbibresource{cv_1}
\nocite{*}


%----------------------------------------------------------------------------------------
%	YOUR NAME AND ADDRESS(ES) SECTION
%----------------------------------------------------------------------------------------
\setlength{\parindent}{0.5cm}
\begin{document}

\name{\textsc{Venkaiah Chowdary Kavuri}}
\vspace{-2mm}
\begin{center} 
\begin{footnotesize}
\contact{4404 Walnut St., \# 3R}{Philadelphia}{USA 19104}{\Email venk@sas.upenn.edu}{\Mobilefone\, (814) 431-2033}
\end{footnotesize}

\href{https://www.linkedin.com/in/venkaiahchowdarykavuri}{\faLinkedinSquare} \href{https://github.com/Venki-Kavuri}{\faGithub} \href{https://scholar.google.com/citations?hl=en&user=r5E9ACIAAAAJ&view_op=list_works}{\aiGoogleScholar} \href{https://www.facebook.com/venki.kavuri}{\faFacebookOfficial}
\end{center} 
\vspace{5mm} 
\today\\
\vspace{5mm} 



Dear Hiring Manager,\\
\vspace{2mm}

I am Venkaiah Kavuri, postdoctoral researcher from Dr. Arjun Yodh\textquotesingle s lab in Department of Physics at University of Pennsylvania. I am writing in to express my interest in \textbf{Engineer/Scientist} position at your company. As a forward thinking postdoc, I am motivated to be part of several projects simultaneously by involving in different roles such as development, data processing/programing and testing. Areas of strength that will guide me in delivering successes towards meeting your organization's goals include  \textbf{Optical Instrument Development, Signal Processing, Image reconstruction, Algorithm Development, and Statistical Analysis.}\\
\vspace{2mm} 


Some of my technical proficiencies include:\\
\begin{itemize}\itemsep -2pt
\item Designing, engineering and testing of optical device prototypes and translating them to clinics.\\
\item Demonstrated record in optical modeling using problem-solving techniques such as Numerical modeling/Simulation, Imaging-Inverse problems, Imaging-Optimization and Data mining.\\

\end{itemize} 

While working as a Postdoc at University of Pennsylvania; I designed instruments to sense/measure changes in optical absorption, scattering and dynamic scattering. I also posses strong fundamental knowledge about amplitude, phase, polarization and coherence properties of NIR-light and using them to probe information about biological tissue. Specifically I originated, designed and built an optical flow and absorption measurement device to be used in neuro-ICU for predicting recurrent stroke. \\
I am also working on wearable blood flow measurement device(using Intel Edison) to be used for measuring blood flow in muscle. During my Phd at University of Texas Arlington, I applied statistical methods to develop a new technique for detecting and imaging early stages of aggressive prostate cancer by creating, engineering and building a TRUS-compatible DOT Probe. I also originated, designed and built multi-spectral, low-cost and portable DOT instrumentation then improved existing image reconstruction methods by combining depth compensation algorithm and L1-regularized least squares method.\\
\vspace{2mm} 
Accomplishing large scale projects offered opportunities to take an active leadership and team member roles for work within an interdisciplinary team of scientists and physicians. I put together (hiring/recruiting) and directly supervised a team to help me with developing the device and work with patients. I welcome the opportunity to participate in a personal interview to answer your questions and better present my qualifications. I look forward to speaking with you soon.\\
\vspace{2mm} 
Thank you for your time and consideration.\\
\vspace{10mm} 
Sincerely,\\

\vspace{5mm} 
Venkaiah Kavuri\\



\end{document}